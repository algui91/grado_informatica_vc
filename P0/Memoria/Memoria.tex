%%% LaTeX Template: Two column article
%%%
%%% Source: http://www.howtotex.com/
%%% Feel free to distribute this template, but please keep to referal to http://www.howtotex.com/ here.
%%% Date: February 2011

%%% Preamble
\documentclass[	DIV=calc,%
							paper=a4,%
							fontsize=11pt]{scrartcl}	 					% KOMA-article class

\usepackage[spanish]{babel}										% English language/hyphenation
\usepackage[protrusion=true,expansion=true]{microtype}				% Better typography
\usepackage{amsmath,amsfonts,amsthm}					% Math packages
%\usepackage[pdftex]{graphicx}									% Enable pdflatex
\usepackage[svgnames]{xcolor}									% Enabling colors by their 'svgnames'
\usepackage[hang, small,labelfont=bf,up,textfont=it,up]{caption}	% Custom captions under/above floats
\usepackage{epstopdf}												% Converts .eps to .pdf
\usepackage{subfig}													% Subfigures
\usepackage{booktabs}												% Nicer tables
\usepackage{fix-cm}													% Custom fontsizes
\usepackage{hyperref}

\usepackage[T1]{fontenc}
\usepackage[utf8]{inputenc}
\usepackage[light,math]{kurier}


%%% Custom sectioning (sectsty package)
\usepackage{sectsty}													% Custom sectioning (see below)
\allsectionsfont{%															% Change font of al section commands
	\usefont{T1}{mdugm}{b}{it}%										% bch-b-n: CharterBT-Bold font
	}

\sectionfont{%																% Change font of \section command
	\usefont{T1}{mdugm}{b}{it}%										% bch-b-n: CharterBT-Bold font
	}

%%% Headers and footers
\usepackage{fancyhdr}												% Needed to define custom headers/footers
	\pagestyle{fancy}														% Enabling the custom headers/footers
\usepackage{lastpage}

% Header (empty)
\lhead{}
\chead{}
\rhead{}
% Footer (you may change this to your own needs)
\lfoot{\footnotesize \miit{Alejandro Alcalde Barros} \textbullet ~}
\cfoot{}
\rfoot{\footnotesize Página \thepage\ de \pageref{LastPage}}	% "Page 1 of 2"
\renewcommand{\headrulewidth}{0.0pt}
\renewcommand{\footrulewidth}{0.4pt}



%%% Creating an initial of the very first character of the content
\usepackage{lettrine}
\newcommand{\initial}[1]{%
     \lettrine[lines=3,lhang=0.3,nindent=0em]{
     				\color{DarkGoldenrod}
     				{\textsf{#1}}}{}}



%%% Title, author and date metadata
\usepackage{titling}															% For custom titles

\newcommand{\HorRule}{\color{DarkGoldenrod}%			% Creating a horizontal rule
									  	\rule{\linewidth}{1pt}%
										}

\pretitle{\vspace{-30pt} \begin{flushleft} \HorRule
				\fontsize{50}{50} \usefont{T1}{kurier}{l}{it} \color{DarkRed} \selectfont
				}
\title{VC P0: Introducción a OpenCV}					% Title of your article goes here
\posttitle{\par\end{flushleft}\vskip 0.5em}

\preauthor{\begin{flushleft}
					\large \lineskip 0.5em \usefont{T1}{mdugm}{m}{it} \color{DarkRed}}
\author{Alejandro Alcalde,}											% Author name goes here
\postauthor{\footnotesize \usefont{OT1}{mdugm}{m}{it} \color{Black}
					Universidad de Granada 								% Institution of author
					\par\end{flushleft}\HorRule}

\date{\usefont{T1}{mdugm}{b}{it}\selectfont\today}																				% No date

\newcommand{\miit}[1]{{\usefont{T1}{mdugm}{m}{it}\selectfont #1}}

\usepackage{minted}

\newminted{bash}{
		numbersep=5pt,
		autogobble=true,
		frame=lines,
		framesep=2mm,
		fontsize=\scriptsize,
		tabsize=2,
		fontfamily=DejaVuSansMono-TLF,
}

\newminted{cpp}{
		numbersep=5pt,
		autogobble=true,
		frame=lines,
		framesep=2mm,
		fontsize=\scriptsize,
		tabsize=2,
		fontfamily=DejaVuSansMono-TLF,
		breaklines,
}

\newmintedfile[myCpp]{cpp}{
    linenos,
    numbersep=5pt,
		autogobble=true,
    frame=lines,
    framesep=2mm,
    fontsize=\scriptsize,
		tabsize=2,
		fontfamily=DejaVuSansMono-TLF,
}

\newcommand{\cppf}[2]{
    \myCpp[label=#2]{#1}
}

\newmintinline{cpp}{fontsize=\scriptsize, fontfamily=DejaVuSansMono-TLF}

%%% Begin document
\begin{document}
\maketitle
\thispagestyle{fancy} 			% Enabling the custom headers/footers for the first page
% The first character should be within \initial{}

\section{Questión 1}

\miit{¿Qué relación hay en OpenCV entre imágenes y
matrices? Justificar la respuesta}\\

Las matrices representan la información de una imagen. Son un \textit{array} numérico n-dimensional de uno (Blanco y negro) o varios (color) canales. Pueden declararse de varios tipos, si la imagen es en blanco y negro, podría usarse una representación \cppinline/unigned int/  de un canal para imágenes en blanco y negro. Ésta representación viene dada en \textit{OpenCV} por la constante \cppinline/CV_8UC1/. Una matriz declarada de este modo representa una imagen en la que cada pixel está formado por un \cppinline/unsigned int/ (8 bits), y un canal (Blanco y negro).

\section{Questión 2}

\miit{Diga el significado de los siguientes tipos OpenCV:
8UC1, 8UC2, 8UC3, 32SC1, 32SC2, 32SC3, 32FC1,
32FC2, 32FC3. ¿Cuáles de ellos están asociados a
imágenes? Justificar la respuesta.}\\

\begin{itemize}
	\item[\cppinline/8UC1/:] Los elementos de la matriz son 8 bits \cppinline/unsigned/, con un solo canal.
	\item[\cppinline/8UC2/:] Elementos de 8 bits \cppinline/unsigned/, dos canales.
	\item[\cppinline/8UC3/:] Elementos de 8 bits \cppinline/unsigned/, tres canales.
	\item[\cppinline/32SC1/:] Elementos de 32 bits \cppinline/signed/, un canal.
	\item[\cppinline/32SC2/:] Elementos de 32 bits \cppinline/signed/, dos canales.
	\item[\cppinline/32SC3/:] Elementos de 32 bits \cppinline/signed/, tres canales.
	\item[\cppinline/32FC1/:] Elementos de 32 bits \cppinline/float/, un canal.
	\item[\cppinline/32FC2/:] Elementos de 32 bits \cppinline/float/, dos canales.
	\item[\cppinline/32FC3/:] Elementos de 32 bits \cppinline/float/, tres canales.
\end{itemize}


En principio todas podrían asociarse con imágenes. La diferencia entre los distintos tipos radicaría esencialmente en el canal. 1 canal para blanco y negro, 2 para escala de grises y 3 para color. Y la profundidad de color (8 bits, 16 bits y 32.)

\section{Questión 3}

\miit{¿Qué relación existe entre cada tipo visual de una
imagen: (color, grises, blanco y negro) y los tipos de
almacenamiento de OpenCV ? Justificar la respuesta}\\

Comencemos por las imágenes en blanco y negro. Para dichas imágenes únicamente es necesario un canal, ya que un pixel podrá ser o blanco o negro. En cuanto a su representación en opencv, bastaría con tener un tipo \cppinline/2UC1/ o algo parecido, ya que solo necesitamos un bit para cada pixel. Un pixel negro sería el 0, blanco el 1, con un solo canal. Dado que este tipo no existe, una imagen en blanco y negro podría representarse en OpenCV con los tipos \cppinline/8UC1, 32SC1, 32FC1/.

Las imágenes a escala de grises se representan usando dos canales, aquí en lugar de tener dos posibles valores, se dispone de un rango de valores desde el blanco al negro. Los tipos que representan este tipo de imágen son  \cppinline/8UC2, 32SC2, 32FC2/.

Por último, las images a color necesitan tres canales (RGB) para representar el color del píxel. El color del pixel se obtiene mediante la combinación de los niveles de {\color{red} Red}, {\color{green} Green}, y {\color{blue} Blue}. Los tipos que se pueden usar en OpenCv para este tipo de imágenes son \cppinline{8UC3, 32SUC3, 32FC3}.

\section{Questión 4}

\miit{¿Es posible realizar operaciones entre imágenes de
distinto tipo visual? Justificar la respuesta}\\

No. Ya que la estructura interna de la matriz no coincide. No se puede por ejemplo, sumar un elemento de una imagen de 32 bits y 3 canales con otra de 8 bits y un canal. Sí que se puede convertir una de ellas al tipo de la otra, y entonces operar. Por ejemplo, dadas una \cppinline/Mat/ de \cppinline/8UC1/ y otra de \cppinline/32SC3/. Si copiamos una a la otra, con \cppinline/copyTo()/, se trasformarán los tipos adecuadamente.

\section{Questión 5}

\miit{¿Cuál es la orden OpenCV que permite transformar el
tipo de almacenamiento de una matriz en otro distinto?}\\

La función \cppinline/void cv::Mat::convertTo()/ es la más indicada para este trabajo, aunque es posible cambiar el tipo de la matriz también con \cppinline/create/. Veamos a continuación cómo podríamos  usarlas, supongamos que tenemos una matriz \cppinline/m/ de tipo \cppinline/8UC1/ declarada \cppinline/Mat m(100, 100, CV_8UC1)/ y queremos pasarla a \cppinline/8UC3/:

Con \cppinline/convertTo()/ bastaría hacer \cppinline/m.convertTo(m, CV_8UC3)/ y con \cppinline/create/ haríamos \cppinline/m.create(100, 100, CV_8UC3)/.

\section{Questión 6}

\miit{¿Cuál es la orden OpenCV que permite transformar el
tipo visual de una imagen en otro distinto? ¿Por qué es
distinta de la que transforma un tipo de almacenamiento
en otro?}\\

La función es \cppinline/cvtColor/. Digamos que es una función auxiliar de creación de Matrices, las que hemos visto hasta ahora (\cppinline/create y convertTo/) son más generales. A \cppinline/cvtColor/ se le pasa como parámetro la matriz/imagen origen, la destino y el tipo al que se quiere comvertir (RGB, Grayscale etc), internamente \cppinline/cvtColor/ se encargará de llamar a \cppinline/create/ con el nuevo tipo. Si se investiga el código de esta función, se puede observar que efectivamente es así (\cppinline/_dst.create( sz, CV_MAKETYPE(depth, dcn))/). Donde \cppinline/_dst/ es la matriz de destino.


\section{Ejercicio 1}

\miit{Escribir una función que lea una imagen en niveles de gris o en
color ( im=leeimagen(filename, flagColor))}\\

\begin{cppcode}
/**
 * Reads an image in GrayScale or color
 *
 * @param name name Name of the image to read
 * @param flag If set to 0, reads the image in color, otherwise in grayscale, default 0.
 * @return A @ref cv::Mat matrix representing the image
 */
Mat leerImagen(std::string name, int flag) {
   return imread(name, flag ? IMREAD_GRAYSCALE : IMREAD_COLOR);
}

Mat a = leerImagen("lena.jpg", 0);
\end{cppcode}

Esta función simplemente es un \textit{wrapper} de \cppinline{imread}, que leerá la imagen en escala de grises o color, segun el \textit{flag} pasado como parámetro.

\section{Ejercicio 2}

\miit{Escribir una función que visualice una imagen (pintaI(im))}\\

\begin{cppcode}
/**
* Shows an image in the screen
*
* @param m Image to show
* @param wn Name of the window
*/
void pintaI(Mat &m, std::string windowName){
    if (!m.empty()){
        namedWindow(windowName, WINDOW_AUTOSIZE);
        imshow(windowName, m);
        waitKey(0);
        destroyWindow(windowName);
    }
}

pintaI(a, "MiVentana");
\end{cppcode}

El propósito de \cppinline/pintaI/ es encapsular el proceso de mostrar una imagen por pantalla. Se asigna un nombre a la ventana, se crea y se muestra. Se espera a que el usuario pulse una tecla y se destruye.

\section{Ejercicio 3}

\miit{Escribir una función que visualice varias imágenes a la vez:
pintaMI(vim). (vim será una secuencia de imágenes) ¿Qué pasa
si las imágenes no son todas del mismo tipo: (nivel de gris, color,
blanco-negro)?}\\

No es posible pintarlas si no son del mismo tipo, ya que la estructura interna de la imagen es distinta. Para conseguir mostrarlas, hay que convertirlas al mismo tipo de estructura. En este caso, se ha tomado la decisión de representar la matriz como un \cppinline/CV_8UC3/, de modo que las imágenes a color puedan mostrarse con color. Cualquier imagen distinta a este tipo, será convertida para poder mostrarla por pantalla.

\begin{cppcode}
	/**
	* Shows a list of images
	* @param m list of images to show. If they are of different types, all are
	* converted to color (CV_8UC3)
	*/
	void pintaMI(const std::vector<Mat> &m) {
	    if (!m.empty()) {
	        int height = 0;
	        int width = 0;

	        // Get the size of the resulting window in which to draw the images
	        // The window will be the sum of all width and the height of the greatest image
	        for (std::vector<Mat>::const_iterator it = m.begin(); it != m.end(); ++it) {
	            Mat item = (*it);
	            width += item.cols;
	            if (item.rows > height) {
	                height = item.rows;
	            }
	        }

	        // Create a Mat to store all the images
	        Mat result(height, width, CV_8UC3);

	        int x = 0;
	        for (std::vector<Mat>::const_iterator it = m.begin(); it != m.end(); ++it) {
	            Mat item = (*it);
	            // If a image is in grayscale or black and white, convert it to 3 channels 8 bit depth
	            if (item.type() != CV_8UC3) {
	                cvtColor(item, item, CV_GRAY2RGB);
	            }
	            Mat roi(result, Rect(x, 0, item.cols, item.rows));
	            item.copyTo(roi);
	            x += item.cols;
	        }
	        pintaI(result, "Ventana");
	    }
\end{cppcode}



\section{Ejercicio 4}

\miit{Escribir una función que modifique el valor en una imagen de una
lista de coordenadas de píxeles.}\\

\end{document}
